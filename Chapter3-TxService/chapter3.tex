\chapter{AbaaS - Atomic Broadcast as a Service}

% **************************** Define Graphics Path **************************
\ifpdf
    \graphicspath{{Chapter3/Figs/Raster/}{Chapter3/Figs/PDF/}{Chapter3/Figs/}}
\else
    \graphicspath{{Chapter3/Figs/Vector/}{Chapter3/Figs/}}
\fi

%Introduction

This chapter introduces the concept of providing \emph{abcast} messaging as a service (\textsf{AbaaS}) to members of a cluster.

First we describe the rational behind \textsf{Abaas}, then we explore the requirements of such a service and the challenges involves in meeting them.  This is followed by the introduction of the \textsf{ABService} protocol that is used to implement \textsf{AbaaS}.  We then explain the methodology used to evaluate \textsf{ABService}, before presenting the results of our performance evaluation.  Finally, we discuss the limitations of existing \emph{abcast} protocols in the context of \textsf{AbaaS}, and propose the need for a new non-blocking \emph{abcast} solution.  

\section{Rational}
% Reduce N-1 communication, by utilising a small set of dedicated nodes to provide abcast for a cluster
% N-1 does not scale
% No overlapping subsets of destinations, fixed set, therefore always faster
% Allows for two optimisations that are not feasable in a traditional abcast solution:
	% Client Abcast requests can be bundled, regardless of m.dst. 
	% Single destination set allows for acknowledgments to be piggybacked, allowing for 1-phase abcast

\section{AbaaS System Requirements}
% Split into client and s-node requirements
	% Client Requirements:
		% What the Abaas System needs to provide consunmers
		% Ordering, previous orderings
		% Must be able to provide abcasts to multiple disjoint destination sets within a cluster of clients
		
	% S-node Requirements:
		% Consensus needs to be reached between s-node members in order to replicate abcast ordering returned to clients
		% S-nodes need to exchange as little data as possible to improve ordering time?????

\section{Abaas Protocol}
% Detail the protocol that I created
	% How it handles cascading orderings
% Explain the various optimisations that are possible
% Utilise diagrams to explain the architecture

\section{Experimentation}
% Explain - 10 clients, x box members etc.  Utilise Aramis paper
% Flow control used in the box and in traditional TOA.  

\section{Results}
% Utilise the graphs used in the Aramis paper, but without the Aramis results
% These graphs can then be used later, with the addition of the Aramis results in order to compare performance

\section{Limitations of Existing Atomic Broadcast Solutions}
% TOA blocks, which if a crash occurs inside the Abaas results in the entire cluster blocking
% Need a new non-blocking protocol that still allows high throughput i.e. not chubby, or zookeeper