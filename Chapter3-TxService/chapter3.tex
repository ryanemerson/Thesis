\chapter{AmaaS - Atomic Multicast as a Service}

% **************************** Define Graphics Path **************************
    \graphicspath{{Chapter3-TxService/Figs/Vector/}{Chapter3-TxService/Figs/}}

This chapter introduces the concept of providing \emph{amcast} messaging as a service to members of a cluster.

First we describe the rational behind \textsf{Amaas}, then we explore the requirements of such a service and the challenges involved in meeting them.  This is followed by the introduction of \textsf{SCast} - an \emph{amcast} protocol that utilises the \textsf{AmaaS} model.  Finally, we discuss the limitations of existing \emph{abcast} protocols in the context of an \textsf{AmaaS} ordering service, and propose the need for a new non-blocking \emph{abcast} solution.  

\section{Rational}
Total order commit protocols can be utilised by distributed systems to coordinate transactions without the use of locks.  Reducing the abort rate of transactions when contention is high, as system deadlocks cannot occur when distributed locks are not present.  Therefore, total order commit protocols can aid scalability as they improve transaction throughput \citep{Ruivo:2011:ETO:2120967.2121604}.  

The limiting factor of total order commit protocols is the underlying protocol used to provide atomic guarantees on message delivery.  The \emph{amcast} protocol, TOA, currently utilised by Infinispan, does not scale well as the number of destinations $N$ increase, as $N->1$ communication is expensive ($\S$ \ref{ssec:TOA_limations}).  Similarly, other GM protocols such as Newtop \citep{Ezhilchelvan:1995:NFG:876885.880005}, exasperate the problem, as the number of messages required to perform an \emph{amcast} increases dramatically as $N$ increases.  Finally, quorum-based protocols provide even less scalability, then GM based protocols, as their inability to \emph{amcast} messages to disjoint sets of nodes typically requires all nodes in the cluster to participate in an \emph{abcast}.  

Regardless of the protocol used, $N->1$ communication is inherently unscalable.  Therefore, we propose that reaching a consensus on ordering should not be conducted between a transaction coordinator $Tx.c$ and its participating Infinispan nodes $Tx.dst$.  Instead consensus should be restricted to an independent coordination service, consisting of a dedicated set of nodes, that agree upon a global ordering for a transaction, before multicasting the $Tx$ to all $d \in Tx.dst$.  We call this system model Atomic Multicast as a Service (\textsf{AmaaS}), and refer to the existing Infinispan approach as \emph{peer-to-peer} (P2P).  

Utilising \textsf{AmaaS} restricts the number of nodes required to reach consensus on a transaction's ordering to the nodes providing the \emph{amcast} service; known as \emph{service} nodes, or simply $s$-nodes, and denoted as $N_s$; Infinispan nodes are known as \emph{client} nodes, or simply $c$-nodes, and denoted as $N_c$.  All $N_s$ in the multicast service utilise the state machine approach, with each $s$-node forwarding their received $c$-node requests to all other $s$-nodes in the service.  An \emph{abcast} protocol is utilised by the $s$-nodes to forward $c$-node requests to all $s$-nodes, ensuring that all of the state machines ($s$-nodes) process $c$-node requests in the same order, therefore guaranteeing that the distributed state of $s$-nodes remains consistent.  

Figure \ref{fig:abaas_concept} shows the three key stages in the \textsf{AbaaS} model.  Stage 1 involves the $Tx.c$ requesting an ordering for $Tx$, $req(Tx)$,  from the ordering service, and occurs once the $Tx.c$ has been executed locally and is ready for multicasting to each destination $d \in Tx.dst$.  Upon receiving $req(Tx)$, the multicast service updates the records of all $s$-nodes, and sends a total order  value for $Tx$ back to $Tx.c$ (Stage 2), whom upon receiving the ordering, multicasts $Tx$ to all $d \in Tx.dst$ (Stage 3).  


    \begin{figure}[htbp!] 
        \centering    
         \includegraphics[width=1.0\textwidth]{amaas_concept}
         \caption[Atomic Multicast as a Service Concept Diagram]{Atomic Multicast as a Service Concept Diagram}
         \label{fig:abaas_concept}
    \end{figure}	 

Decoupling message multicasting and ordering means that regardless of $\left\vert Tx.dst \right\vert$, the number of nodes that need to reach a consensus in order to obtain a total order for $Tx$ will always be equal to $\left\vert s\text{-nodes}\right\vert$.  Restricting consensus to $s$-nodes means that, for all transactions the nodes involved in the consensus process will always be the same, therefore various optimisations can be made to the \emph{abcast} protocol used for consensus ($\S$ \ref{ssec:atomic_broadcast}), in addition to several other optimisations that can be made to the service itself ($\S$ \ref{ssec:abaas_optimisations}).  

Figure \ref{fig:ordering_service_concept} shows how an ordering request is handled by the $s$-node that receives an ordering request.  Stage 1 is simply $N_s1$ receiving the request $req(Tx)$ from $Tx.c$.  Upon receiving $req(Tx)$, $N_s1$ \emph{abcast}s the request to all $s$-nodes including itself (Stage 2).  Once $N_s1$ has received its own \emph{abcast} message, it is possible for $req(Tx)$ to be processed and a total order assigned to $Tx$, before a response $Rsp(Tx)$ is sent back to $Tx.c$ containing the total order data (Stage 3).  Note, that regardless of the number of destinations involved in $Tx$, the number of $s$-nodes participating in the \emph{abcast} is constant.  


    \begin{figure}[htbp!] 
        \centering    
         \includegraphics[width=1.0\textwidth]{ordering_service_concept}
         \caption[Multicast Service Concept Diagram]{Multicast Service Concept Diagram}
         \label{fig:ordering_service_concept}
    \end{figure}	 

	Lastly, implementing transaction ordering as a dedicated service, provides distinct advantages for system administrators.  When utilising the P2P \emph{amcast} approach, administrators would typically want to run Infinispan over a cluster of nodes utilising a homogeneous hardware specification to ensure that Infinispan's performance would not be hindered by a lesser machine.  In an environment where low-latency and high-capacity in-memory database is desired, this would require a large number of expensive machines. However, when \textsf{AmaaS} is utilised, it is possible to improve the performance of the entire cluster, simply by upgrading the $s$-nodes used to provide the \textsf{AmaaS} service.  For example, consider a cluster consisting of $50$ $c$-nodes, and $3$ $s$-nodes, instead of upgrading all $50$ $c$-nodes it is possible to improve transaction latency and throughput by upgrading the hardware of the $3$ $s$-nodes.  
	
	\subsection*{AmaaS Unique Optimisations}\label{ssec:abaas_optimisations}
	The \textsf{Amaas} model allows for two key optimisations that are not possible when \emph{amcast}ing occurs directly between $c$-nodes in the P2P approach: \emph{Message Bundling} and \emph{Acknowledgement Piggybacking}.  
	
		\subsubsection*{Message Bundling}
		%\paragraph{Message Bundling} \hspace{0pt} \\
		As the number of concurrent transactions between $c$-nodes increases, the total number of \emph{amcast}s required also increases.  When utilising \emph{amcast}s between $c$-nodes to coordinate the transactions, typically it is not possible to bundle multiple \emph{amcast} messages $<m_i, m_j>$, into a single \emph{amcast}, $m$, due to the high probability of $m_i.dst \neq m_j.dst$.  Of course it is possible to implement a bundling strategy that is utilised only when $m_i.dst = m_j.dst$, however in a system such as Infinispan the performance improvements provided by such a strategy are negligible; as the wide distribution of key/value pairs significantly reduces the probability of two \emph{amcast}s having the same destination set.  
		
		When utilising \textsf{AmaaS} it is possible for all \emph{amcast} requests received from $c$-nodes to be bundled into a single \emph{abcast} (between $s$-nodes) at a receiving $s$-node, regardless of their destination set.  This is because $s$-nodes are only required to send \emph{ambast}s to other $s$-nodes in order for a consensus on transaction ordering to be reached, therefore the destination set for each \emph{abcast} is the same for all $c$-node requests.   The ability to bundle multiple \emph{amcast} requests into a single \emph{abcast} reduces the number of times that consensus needs to be reached between all $s$-nodes.  Thus further reducing the number of $N->1$ communication steps required, with the total number of \emph{abcast}s reduced by $\left\vert bundle \right\vert$; where $bundle$ is the number of  \emph{amcast} requests from $c$-nodes that are sent as a single \emph{abcast}.  As a result of this optimisation network traffic is significantly reduced when requests are frequent, resulting in the capacity and scalability of an \emph{AmaaS} service increasing. Conversely, message bundling does not compromise performance when the number of service requests is low, as bundling does not require any intensive computation or additional communication steps.  
		
		\subsubsection*{Acknowledgement Piggybacking}
		%\paragraph{Acknowledgement Piggybacking} \hspace{0pt} \\
		In the \textsf{AmaaS} model, all executions of the \emph{abcast} protocol used for consensus, occur between the same set of $s$-nodes in the \emph{multicast service}, and hence, every \emph{abcast} has the same destination set.  As each \emph{abcast} is guaranteed to be received by all $s$-nodes it is possible for message acknowledgements to be piggybacked, therefore enabling \emph{abcast}s to be satisfied with a single dedicated broadcast ($\S$ \ref{ssec:newtop}).  This optimisation is ideal for \emph{abcast}s between $s$-nodes, as all nodes in the service must handle $c$-node requests, therefore each $s$-node will be sending \emph{abcast}s frequently, allowing message acknowledgements to be piggybacked, and reducing the average number of messages required per \emph{abcast}.  	
	
\section{Limitations of Existing Coordination Services}\label{sec:limitations_existing_coordination}
It is possible to utilise existing coordination solutions ($\S$ \ref{sec:coordination}), such as Zookeeper\citep{Hunt:2010:ZWC:1855840.1855851} and Chubby\citep{Burrows:2006:CLS:1298455.1298487}, as the basis of a \textsf{AmaaS} service.  However, both of these solutions are intended for high levels of read requests, not workloads that consist predominantly of write operations; where a write operation is any client operation that changes the state of a single $s$-node, therefore requiring the state of all other $s$-nodes be updated in order to maintain a consistent state machine.  

Write operations are vital to the \textsf{AmaaS} model, as every ordering request from a client requires an update to an $s$-nodes records and hence requires a consensus to be reached between $s$-nodes.  Both Zookeeper and Chubby propagate write operations by utilising a quorum based \emph{abcast} protocol to obtain a consensus.  Consequently, all write operations are coordinated by a single master node, and as a result the throughput of such a service is always be limited by the capabilities of the elected master node.  For example, the Chubby service utilises an implementation of the Paxos protocol to coordinate write operations.  However, the performance of paxos and its variants have been shown to be unpredictable under even mild levels of stress, with sustained periods of demand preventing a consensus being reached \citep{DBLP:journals/corr/MarandiBPB14}.  Therefore, if either of the Quorum based implementations Zookeeper or Chubby were utilised as the basis of an \textsf{AmaaS} service, the service's performance would be poor due to the larger number of write operations required.  The observed limitations of these existing solutions was the motivation for requirement S4 in $\S$ \ref{sec:absaas_requirements}.  	
	
\section{AmaaS Requirements}\label{sec:absaas_requirements}
The \textsf{AbaaS} model consists of two distinct entities: $s$-nodes and $c$-nodes.  This section will explore the requirements that need to be met in order for the \textsf{AmaaS} model to be effective.  We consider requirements from the perspective of both clients, $c$-nodes, and the multicast service, $s$-nodes.

	\subsection*{Client Requirements}
	\begin{itemize}
		\item [\textbf{CR1}] Clients must be able to send \emph{amcast}s to multiple destination sets that may overlap.
		
		\item [\textbf{CR2}] The service must provide a consistent total order $m.ts$ on messages irrespective of the $s$-node handling a clients request, or the $c$-node from which the clients request originated.  
		
		\item [\textbf{CR3}] Client nodes must be informed of $m_i$ when handling $m_j$, if $m_i.dst \cap m_j.dst$ to ensure that $m_i$ is not missed by a $c$-node in $m_i.dst$.  
	\end{itemize}
	
%	\paragraph{Service Requirements} \hspace{0pt}
    \subsection*{Service Requirements}
	\begin{itemize}
		\item [\textbf{S1}] The service must provide fault-tolerance ($s\text{-nodes} > 1$).
		
		\item [\textbf{S2}] The service must be highly available and non-blocking in the event of $s$-node failures.  Necessary to prevent the entire cluster becoming blocked if a $s$-node fails.   
		
		\item [\textbf{S3}] All service nodes must process client requests in the exact same order to maintain a consistent state between all $s$-nodes.  This prevents a $c$-node from receiving conflicting ordering data, for example two distinct $m$ being allocated the same global timestamp.  
		
		\item [\textbf{S4}] All $s$-nodes should be able to handle client requests, to allow for high availability and to prevent a single $s$-node becoming a performance bottleneck.
	\end{itemize}

\section{SCast: Atomic Multicast Protocol for AmaaS}\label{sec:scast_protocol}
We have developed a protocol for the \textsf{AmaaS} model, which we call \textsf{SCast}; as the protocol offers atomic ordering for multicast messages as a service, hence Service Multicast - \textsf{SCast}.  \textsf{SCast} satisfies all of the requirements specified in $\S$ \ref{sec:absaas_requirements} with the exception of S2.  S2 cannot be satisfied by our protocol directly, rather \textsf{SCast} relies on a \emph{abcast} protocol to guarantee S1, S3 and S4, therefore in order for S2 to be satisfied, the underlying \emph{abcast} protocol must not block in the presence of node failures.  

\textsf{SCast} defines how client nodes utilise the ordering information provided by the multicast service and how they interact with the service, as well as how the service nodes handle client requests and maintain their replicated state, therefore \textsf{SCup} is explained from the perspective of both client and service nodes.  We then discuss the fault-tolerance of \textsf{SCup}, exploring how the atomicity of the protocol is maintained in the event of node crashes at various stages of the protocols execution and how the \emph{multicast service} can tolerate split brain partitions.  

In the explanations below we assume that Infinispan is executing a 1-Phase Total Order transaction, without a second WSC phase, and that the transaction has already been successfully executed locally.  Furthermore, we assume that a reliable network protocol is being utilised as the underlying communication mechanism, for example TCP\citep{Cerf:2005:PPN:1064413.1064423} or Reliable UDP\citep{ReliableUDP}.  Finally, we refer to a collection of $s$-nodes providing the \emph{amcast} service as the \emph{multicast service}.  


    \subsection{Protocol Details}
    \subsubsection*{Client Nodes}
    Stage 1 of the client \textsf{SCup} protocol starts once a transaction coordinator, $Tx_i.c$, has completed its local execution of $Tx_i$ it is ready to \emph{amcast} a $prepare(Tx_i)$ message to $Tx_i.dst$ as required by the total order commit protocol.  Each stage of the client protocol and its purpose are detailed below:
    \begin{description}
    
        \item[1. Delegate Backup Coordinator] \hfill \\
        For the purposes of fault-tolerance the first stage of the protocol is for $Tx_i.c$ to create a backup coordinator.  $Tx_i.c$ selects any destination, $d$, $d \in Tx_i.dst$, and sends it the payload of the message ($prepare(Tx_i)$) that is to be multicast to $Tx_i.dst$.  $Tx_i.c$ then waits for an acknowledgement of receipt from $d$ before proceeding to stage 2 of the protocol.  If an acknowledgement is not received within a configurable amount of time, then $Tx_i.c$ simply selects another destination $d'$ from $Tx.dst$ and restarts the process.  
        
        The purpose of creating a backup transaction coordinator is to allow for the possibility that the original $Tx.c$ may crash during the multicast process, in which case the group membership protocol will detect the crash and the backup coordinator will assume responsibility for multicasting the $prepare(Tx_i)$ message.  When the backup coordinator takes over from a crashed coordinator it completely restarts the multicast process and selects another node from $Tx_i.dst$ to be a backup coordinator.  In the unlikely event that subsequent coordinators crash, it is possible that there wont be a member of $Tx.dst$ left to utilise as a backup coordinator.  In which case a backup will not be created, as the current coordinator will be the last destination alive in $Tx_i.dst$, therefore if the current coordinator crashes then the transaction is aborted by default.  
                
        \item[2. Request Ordering from Multicast Service] \hfill \\
        Once the a backup coordinator has been established, it is possible for the transaction coordinator to request a multicast ordering from the service.  \emph{Amcast}s are initiated by the $Tx_i.c$ randomly selecting a $s$-node, $N_s$, from the \emph{ordering service} and sending an \emph{amcast} request to that node - $req(Tx_i) -> N_s$.  If $Tx_i.c$ does not receive an ordering from the \emph{multicast service} after a specified amount of time, then a different $s$-node is selected and the ordering request is resent - $req(Tx_i) -> N_s'$.  
        
        The content of the $prepare(Tx_i)$ message is not sent to $N_s$ as we only require a global order for $Tx_i$, therefore the contents of the message to be \emph{amcast} to $Tx.dst$ is irrelevant and including it would only increase the load on the network.  However, $req(Tx_i)$ must include $Tx_i.dst$ as this is the destination set of the \emph{amcast} that $Tx_i$ is trying to send, and the \emph{multicast service} needs this data to ensure that requirements CR1, CR2 and CR3 are satisfied.  
        
        \item[Waiting for a response from the \emph{multicast service} ...] \hfill \\
        
        \item[3. Receive Ordering Data and Multicast Transaction] \hfill \\
        When the transaction coordinator receives $Rsp(Tx_i)$ from $N_s$, it appends the ordering information to the original $prepare(Tx_i)$ message and sends a multicast, $mcast(Tx_i)$, to all $Tx.dst$ including itself. 
        
        \item[4. Receive Multicast from the Coordinator] \hfill \\
        Upon receiving $mcast(Tx_i)$, a client node, $c$, will check the \emph{immediate} predecessor for this node.  This data dictates which $mcast$ message should be delivered in the total order before $mcast(Tx_i)$.  For example, if the predecessor data dictates $mcast(Tx_i)$ is preceded by $mcast(Tx_h)$ at destination $d$, then $d$ cannot deliver $mcast(Tx_i)$ until it has delivered $mcast(Tx_h)$.  
        
        The predecessor data consists of a single multicast timestamp for each $d \in Tx.dst$, opposed to a list of past timestamps, in order to reduce the size of each $Rsp$ header sent by the \emph{multicast service}.  The use of a single timestamp results in a cascading wait occurring if multiple message have not yet been received by $c$.  For example, if $c$ has received $mcast(Tx_j)$ but has not received its predecessors $mcast(Tx_h)$ and $mcast(Tx_i)$, $c$ will only be aware of $mcast(Tx_i)$ from $Tx_j$'s predecessor data.  However when $mcast(Tx_i)$ arrives, $c$ becomes aware that it has not yet received $mcast(Tx_h)$,  and will wait for $mcast(Tx_h)$ before delivering $mcast(Tx_i)$ and $mcast(Tx_j)$.          
    \end{description}

    \subsubsection*{Service Nodes}
    Stage 1 of the service protocol starts when a multicast request, $req(Tx_i)$ is received from a client node.  Note, it is anticipated that each $s$-node will have to handle many requests simultaneously, however for the sake of brevity we assume the service is only handling a single request.  
    
    \begin{description}
        \item[1. Receive Client Request] \hfill \\
            Upon receiving $req(Tx_i)$, each $s$-node places the request in its \emph{Abcast Request Pool} (ARP), which is a bounded queue for storing requests before they are \emph{abcast} to all $s$-nodes.  If an $s$-node's ARP becomes full, subsequent requests from $c$-nodes are rejected until space becomes available in the ARP.  When a $c$-node request is rejected a \emph{reject} response is sent to $Tx_i.c$.  If $Tx_i.c$ receives a \emph{reject} response from all $s$-nodes, then it can either abort $Tx_i$ or resend the \emph{amcast} request to another $s$-node after a configurable amount of time.    
            
            The ARP is necessary to ensure that if the \emph{ordering service} starts to become overloaded by client requests their is a 'feedback' mechanism that makes clients aware of the services current limitations, allowing clients to restrict user operations if necessary.  Utilising an ARP is also essential for providing message bundling, which as described in \ref{ssec:abaas_optimisations} is an effective optimisation for improving the throughput of the \emph{multicast service}.  
            
        \item[2. Process ARP] \hfill \\
        A single thread, called the \emph{send} thread, is utilised for retrieving requests from the ARP and \emph{abcast}ing them to all $s$-nodes for ordering.  The \emph{send} thread retrieves ordering requests from the ARP in their arrival order, and bundles them into a single message bundle $mb$, before \emph{abcast}ing $mb$ to all $s$-nodes.  If the ARP is empty, then the \emph{send} thread waits for the ARP to become non-empty before resuming \emph{abcast}ing.  
		
		A configurable upper limit is placed on the maximum size of a bundle message. \footnote{The maximum size could be specified in terms of bytes or the number of messages to be bundled.} If this upper limit is reached and the ARP still has available requests, then the \emph{send} thread will \emph{abcast} the next message bundle, $mb'$, once $mb$ has been \emph{abcast}.  If message bundling is not enabled, then a upper limit of one message is set for all bundles.    
		
		All \emph{abcast} $mb$ sent by an $s$-node have an originator field that is set to equal the sending node's address $N_s$, $mb.o$ = $N_s$, this is necessary for the next phase of the protocol.  
		
		\item[3. Process Requests and Return Ordering] \hfill \\
		When an $s$-node, $N_s$, receives a request bundle $mb$, it 'un-bundles' $mb$ and processes each ordering request $req(Tx_i)$ in the order that they arrived in the ARP at $mb.o$.  If $N_s$ has already received $req(Tx_i)$ in a previous \emph{abcast} message, we discard this request and take no further action.  \footnote{Requests can be \emph{abcast} more than once if a client request timesout and the transaction coordinator resends the request to a different $s$-node.} It is possible to discard a repeat request as we know that all other $s$-nodes have, or will eventually, handle(d) the same version of a request as $N_s$ due to the guarantees provided by \emph{abcast}.  A request is accepted by an $s$-node if it is the first time that the $s$-node has encountered the request id.  
		
		Once a request has been accepted it is associated with a global timestamp $ts$: $req(Tx_i).ts = m.ts\oplus m.o \oplus$\emph{sequence number} of $req(Tx_i)$ within the bundle; where $\oplus$ is the append operator and $m.ts$ is the final timestamp provided by the underlying \emph{abcast} protocol utilised between $s$-nodes.  As soon as a timestamp has been assigned to a transaction, it is possible for a response message to be sent to the transaction coordinator.  As all $s$-nodes will receive, and assign the same timestamp to each request, it is necessary for a single $s$-node to take responsibility for sending the response message.  This is achieved by utilising the $mb.o$ field set in the previous stage of the protocol: If $N_s = mb.o$ then $N_s$ is the $s$-node that initially received $req(Tx_i)$, therefore $N_s$ takes responsibility for sending a response message.  
				
		A Response message, $Rsp(Tx_i)$ consists of two types of ordering data: the timestamp agreed by $s$-nodes for $req(Tx_i)$, and the timestamp of $req(Tx_i)$'s \emph{immediate} predecessor.  The latter is the identity of the transaction that directly precedes $Tx_i$ in the total order of multicasts.  More precisely, for all destinations $d \in r(Tx_i).dst$, each $d$ must not deliver $Tx_i$ until they have delivered the transaction whose timestamp is mapped to their address in the immediate predecessor data.  
				
		The storage of \emph{immediate} predecessor data works as follows: All $s$-nodes maintain a map that stores a transaction history by mapping a $c$-node address with the id of the last transaction they were associated with, hence its \emph{immediate} predecessor.  So for each multicast request, the associated timestamp is stored in the map for each destination in the transactions destination set. When a $s$-node receives an \emph{abcast} bundle $mb$, it knows that all other $s$-nodes have received, or will receive, $mb$ in the same order.  Therefore, when $mb$ is processed by an $s$-node, it is guaranteed that all other $s$-nodes will have processed $mb$ in the exact same order, hence we know that the transaction history will be consistent across all $s$-nodes.  
		
		Note that the immediate predecessor of $Tx_i$ is applicable to \emph{all} \emph{amcast}s directed at a given $d$ - not just those that originate from $Tx_i.c$ nor just those that are handled by one $s$-node. Thus, it is specific to each $d \in Tx_i.dst$ and ensures that delivery at every $d$ is per the finalized $Tx.ts$.  To illustrate this, let $Tx_i.c$ submit $req(Tx_i)$ to $N_s$, $Tx_j.c$ submit $req(Tx_j)$ to $N_{s'}$ and $d \in Tx_i.dst \cap Tx_j.dst$. Say, the multicast service places $req(Tx_j)$ before $req(Tx_i)$ in the total order, if $d$ receives $Tx_i$ before $Tx_j$, it cannot deliver $Tx_i$ until it delivers $Tx_j$.
    \end{description}

	\subsection{Fault-Tolerance: Node Crashes}\label{ssec:scast_fault_tolerance}
	Fault-tolerance in \textsf{SCast} must consider the consequences of both crashed $c$-nodes and $s$-nodes.  Here we explore the consequences of both $c$-node and $s$-node crashes during various stages of a \textsf{SCast} \emph{amcast}.  For the sake of simplicity, we only consider node crashes from the perspective of a single transaction, however it is worth noting that each $c$-node would typically have multiple transactions executing concurrently.  
	
    \subsubsection*{Client Node Crash}
	\begin{description}
         \item[\emph{Local Tx Execution}]  \hfill \\
         If a $c$-node, $Tx.c$, crashes during or directly after the local execution of a transaction, $Tx_i$, then no action needs to be taken as no interactions with other $c$-nodes or $s$-nodes have occurred.  
		
		\item[\emph{Stage 1}]  \hfill \\
		If $Tx.c$ crashes during the creation of a backup coordinator then two scenarios are possible:
		    \begin{itemize}
			    \item    The backup coordinator never receives the $prepare(Tx_i)$ message, in which case the transaction  can only be aborted as its contents have been lost.  
			    \item    The backup coordinator successfully receives the $prepare(Tx_i)$ message and attempts to acknowledge the original coordinator.  The original coordinator will never receive the acknowledgement as it has crashed, however the backup coordinator will still restart the multicasting process. The backup coordinator will takeover as soon as the group membership protocol has detected that the original coordinator has crashed.
		    \end{itemize}     
		    
		\item[\emph{Stage 2}]  \hfill \\
        If $Tx.c$ crashes before or during the sending of a request to the multicast service, then the backup coordinator will simply restart the multicast process when the group membership service recognises that $Tx.c$ has crashed.  It is possible that the $s$-node that the request was sent to will still see receive the request, in which case the original request will be processed as normal by the $s$-node.  This means that when the backup coordinator sends a new multicast request to the service, the service will accept the original request from $Tx.c$.  Therefore if a $s$-node receives a request that is from a backup coordinator, not the original $Tx.c$, the $s$-node will record the address of the backup coordinator locally and send a response to the backup coordinator when the original request arrives via \emph{abcast}.  Of course it is possible that the original request was never received, in which case the $s$-node that received the request from the backup coordinator will simply send a response message when the backup coordinators request has been delivered by all $s$-nodes via \emph{abcast}.  
        
        \item[\emph{Stage 3}]  \hfill \\
        If $Tx.c$ crashes before receiving a response from the multicast service, then the backup coordinator takes over and restarts the multicast process.  When the backup coordinators request is received by an $s$-node, the $s$-node checks its recent history of processed requests and returns the ordering data assigned to this transaction.  The size of the recent history stored by the service should be configurable to allow for varying levels of resilience.   This is because as the size of the past history increases, the chances of a prior transaction ordering being discarded decreases.  Thus a larger record provides a greater level of fault-tolerance, but at the expense of utilising more system resources.  
        
        \item[\emph{Stage 4}]  \hfill \\
        If $Tx.c$ crashes before the multicasting of $mcast(Tx_i)$ to all $d \in Tx.dst$ has finished, then two scenarios are possible:
           \begin{itemize}
			    \item    The backup coordinator has received $mcast(Tx_i)$ from $Tx.c$, in which case the backup coordinator can simply restart the multicasting process (Stage 4).
			    \item    The backup coordinator has not received $mcast(Tx_i)$ from $Tx.c$, in which case the protocol must be restarted.  
		    \end{itemize}  
    \end{description}
    
	\subsubsection*{Service Node Crash}
	\begin{description}
       \item[\emph{Stage 1-3}] \hfill \\
       If a $s$-node crashes while a $Tx.c$ is issuing a service request, then the request will simply timeout and the $Tx.c$ will resend the request to another $s$-node.  Its possible that the request was \emph{abcast} to other $s$-nodes before the crash, in which case the other $s$-nodes will process the request and simply return the established ordering for a transaction when subsequent requests are received from the transaction coordinator.  
    \end{description}
    
    \subsection{Fault Tolerance: Split Brain}
    Split brain refers to a situation whereby the current view of a group of processes has been partitioned into two or more views, which is usually caused by one or more failures occurring at the underlying network layer. Typically, these views will be consist of disjoint sets of processes, however it is possible for overlapping to occur between multiple view sets.  Eric Brewer's seminal CAP theorem, states that it is impossible for a system to provide Consistency, Availability and Partition Tolerance simultaneously \citep{Brewer:2000:TRD:343477.343502,6133253, Gilbert:2002:BCF:564585.564601}.  Therefore, when designing a solution for handling split brain scenarios, which is a partition by definition, it is necessary for either availability or consistency of part of the system to be compromised.  
    
    Handling split brain partitions across a cluster in which \textsf{SCast} operates is ultimately the responsibility of the applications using \textsf{SCast} for \emph{amcast}s.  For example, in the case of Infinispan, if a cluster is partitioned then it is the responsibility of Infinispan to determine whether consistency or availability should be preserved.  However, as a Infinispan cluster will be dependent on \textsf{SCast} and a \emph{multicast service}, it is necessary for such a service to provide a strategy for handling partitions that occur within the service itself.  
    
    Our solution to handling partitions is for \textsf{SCast} to utilise a majority partition scheme.  When a network partition occurs, the $s$-nodes whose new network view is a majority of the previous view continues to accept client requests and operate as a \emph{multicast service}.  Whereas the $s$-nodes who are now in the minority partition sacrifice availability by rejecting future client requests until the juncture of the two partitions.  The $s$-nodes in the minority partition reject client requests in order to allow for the consistency of the system to be readily resolved when the two partitions are rejoined.  For example, if the availability of the minority partition was not sacrificed, the merging of state required when the two partitions are rejoined would not be trivial, with the predecessor data and active client requests of each partition having to be fused in a way that does not compromise \emph{amcast} guarantees G1-G4.  Whereas, when only one partition remains active when the network is divided, it is possible for the $s$-nodes in the minority partition to clone the state of an $s$-node from the majority partition and start accepting client requests again.  
    
    The majority partition scheme detailed above works as expected when $|s$-nodes$|$ is an odd number, however if it is an even number, then it is possible for the \emph{multicast service} to be partitioned so that no majority partition exists.  To avoid such a scenario it is necessary for a \textquoteleft{}watcher' node to be utilised when $|s$-nodes$|$ is even.  A watcher node does not participate in the \textsf{SCast} protocol, rather it is used purely for tie-breaking between the views of two $s$-node partitions that would otherwise be equal.  
    
\section{A New Atomic Broadcast Solution is Required}
Existing \emph{abcast} and \emph{amcast} solutions are of two types ($\S$ \ref{sec:atomic_guarantees}); quorum-based and GM based.  GM based protocols provide the lowest latency message delivery in the absence of node crashes, but at the expense of blocking severely when node crashes do occur.  This blocking behaviour is acceptable when such protocols are utilised in traditional P2P environments like Infinispan, as it is presumed that the blocking will only occur at a small subset of nodes in the cluster.  In which case system \emph{liveness} is maintained by the majority of nodes in the cluster.  However in the \textsf{AmaaS} model, if a $s$-node utilises a GM protocol for \emph{abcast}ing requests amongst all $s$-nodes and a single $s$-node crashes, all $s$-nodes will block, resulting in no client requests being satisfied. This means that, not only are the $s$-nodes participating in the \emph{abcast} blocked, but as a consequence of this blocking, so to are all of the $c$-nodes utilising the service.  Therefore the entire system's \emph{liveness} is lost until the GM protocol is able to detect the $s$-node crash and unblock the ordering service.  

Alternatively, a quorum-based protocol, such as those detailed in \ref{sec:coordination}, can be utilised between $s$-nodes.  Such protocols perform worse than GM protocols in the absence of node failures ($\S$ \ref{sec:limitations_existing_coordination}), however they only block mildly when a leader node crashes or is falsely suspected of crashing.  

In order to maximise the effectiveness of the \textsf{AmaaS} system model, a new \emph{abcast} protocol is required.  This protocol must provide non-blocking message delivery in the presence of node failures, whilst allowing for low-latency, high-throughput \emph{abcast}s in their absence.  

\section{Summary}
This chapter presented \textsf{AmaaS} - a new model for \emph{amcast} protocols that utilises a dedicated set of nodes to provide \emph{amcast} as a service to distributed transactional systems.  We then presented a new protocol \textsf{SCast} that provides fault-tolerant \emph{amcast}ing in such an environment.  Lastly, we outlined the shortcomings of existing \emph{abcast} solutions and the need for a new protocol in order for the \textsf{AmaaS} approach to be fully realised. 