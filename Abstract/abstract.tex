% ************************** Thesis Abstract *****************************
% Use `abstract' as an option in the document class to print only the titlepage and the abstract.
\begin{abstract}
Coordinating transactions involves ensuring serializability in the presence of concurrent data accesses. Accomplishing it in an scalable manner for distributed in-memory transactions is the aim of this thesis work. To this end, the work makes three contributions. It first experimentally demonstrates that transaction latency and throughput scale considerably well when an atomic multicast service is offered to transaction nodes by a crash-tolerant ensemble of dedicated nodes and that using such a service is the most scalable approach compared to practices advocated in the literature. Secondly, we design, implement and evaluate a crash-tolerant and non-blocking atomic broadcast protocol, called ABcast, which is then used as the foundation for building the aforementioned multicast service. 

ABcast is a hybrid protocol, which consists of a pair of primary and backup protocols executing in parallel.  The primary protocol is a deterministic atomic broadcast protocol that provides high performance when node crashes are absent, but blocks in their presence until a group membership service detects such failures.  The backup protocol, Aramis, is a probabilistic protocol that does not block in the event of node crashes and allows message delivery to continue post-crash until the primary protocol is able to resume. Aramis design avoids blocking by assuming that message delays remain within a known bound with a high probability that can be estimated in advance, provided that recent delay estimates are used to (i) continually adjust that bound and (ii) regulate flow control. Aramis delivery of broadcasts preserve total order with a probability that can be tuned to be close to 1. Comprehensive evaluations show that this probability can be $99.99\%$ or more. 

Finally, we assess the effect of low-probability order violations on implementing various isolation levels commonly considered in transaction systems. These three contributions together advance the state-of-art in two major ways: (i) identifying a service based approach to transactional scalability and (ii) establishing a practical alternative to the complex PAXOS-style approach to building such a service, by using novel but simple protocols and open-source software frameworks.
\end{abstract}