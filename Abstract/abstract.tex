% ************************** Thesis Abstract *****************************
% Use `abstract' as an option in the document class to print only the titlepage and the abstract.
\begin{abstract}
Coordinating transactions involves ensuring serializability in the presence of concurrent data accesses. Accomplishing it in an elastically scalable manner for distributed in-memory transactions is the aim of this thesis work. To this end, the work makes three contributions. It first experimentally demonstrates that transaction latency and throughput scale considerably well when an atomic multicast service is offered to transaction nodes by a crash-tolerant ensemble of dedicated nodes and that using such a service is the most scalable approach compared to practices advocated in the literature. Secondly, we design, implement and evaluate a crash-tolerant and non-stop atomic broadcast protocol, called ABCast, which is then used as the foundation for building the aforementioned multicast service. 

ABCast is a diversely designed pair of primary and backup protocols that work in tandem seamlessly. The primary protocol is a decentralised version of Skeen, chosen for its high performance that is indifferent to false suspicions in the absence of crashes; when it awaits, post-crash, JGroups to ascertain the crash using conservative timeouts and then deliver a new membership view, the second protocol, called Aramis that operates as a hot backup, takes over to ensure continued availability of the multicast service. Aramis design avoids the FLP impossibility by assuming that message delays remain within a known bound with a high probability that can be estimated in advance, provided that recent delay estimates are used to (i) continually adjust that bound and (ii) regulate flow control. Aramis delivery of broadcasts preserve total order with a probability that can be tuned to be close to 1. Comprehensive evaluations show that this probability can be $99.99\%$ or more. 

Finally, we assess the effect of low-probability order violations on implementing various isolation levels commonly considered in transaction systems. We assume infinispan implementation framework and show that all these levels, including 1-copy serializability, can be preserved.  These three contributions together advance the state-of-art in two major ways: (i) identifying a service based approach to transactional scalability and (ii) establishing a practical alternative to the complex, PAXOS-style approach to building such a service by using novel but simple protocols and open-source software frameworks, such as JGroups.  
\end{abstract}