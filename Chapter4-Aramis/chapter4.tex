\chapter{Aramis}

% **************************** Define Graphics Path **************************
%\ifpdf
%    \graphicspath{{Chapter3/Figs/Raster/}{Chapter3/Figs/PDF/}{Chapter3/Figs/}}
%\else
    \graphicspath{{Chapter4/Figs/Vector/}{Chapter4/Figs/}}
%\fi

This chapter introduces a non-blocking \emph{abcast} protocol, called \textsf{Aramis}, that utilises probabilistic guarantees for message ordering to prevent message delivery blocking in the event of node failures.  First we introduce the rational behind our design approach for \textsf{Aramis}, before detailing the protocol's requirements and assumptions.  This is followed by an in-depth look at the components required by \textsf{Aramis}, and how they have been implemented.  We then explore the protocol in detail, outlining the protocols delivery and rejection criteria for \emph{abcast} messages.  The experiments used to evaluate \textsf{Aramis} are then detailed, followed by an evaluation of the protocol's performance and the ramifications of our findings.  

\section{Rational}
	% AbaaS requires a non-blocking abcast protocol.  
	% FLP impossibility states that it is impossible for a protocol to be non-blocking and 100% deterministic.
	% Relax abcast guarantees in order to allow non-blocking performance. 
	
\section{Assumptions}
	% N - 1 Crashes
	% Reliable UDP utilised
	% Clock synchronisation
	% Stationary Assumption

	\subsection{Probing Validation}
		% Describe Experiment and execution environment
		% Results
		% Analysis
	
\section{Aramis Components}
	\subsection{Network Measurement Component}
	\subsection{Reliable Multicast}
	\subsection{Clock Synchronisation}
	\subsection{Flow Control}
	
\section{Protocol}
	\subsection{Delivery Delay}
		% Maths and the rational for their calculation. 
	\subsection{Rejected Messages}
		% Messages rejected if a later message has already been delivered.
		% Issue a warning, or throw exception.
	
\section{Performance Evaluation}
	% Ran in the emulated transactions environment, very slow performance - due to large upper bound on delivery delays.
	% Not suitable for use as a standalone abcast protocol.
	% However, no rejections occurred - assumptions held.
	
\section{Summary}